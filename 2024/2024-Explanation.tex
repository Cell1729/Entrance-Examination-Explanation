\documentclass{ltjsarticle}
\usepackage{amsmath}
\usepackage{amssymb}
\begin{document}
\title{令和6年度(2024年度) 追試験解説}

\section*{問1}
次の計算をした結果として正しいものを、それぞれ1~4の中から選びなさい。

\begin{enumerate}
    \item $7 - (-12)$
    \begin{enumerate}
        \item $-19$ \quad \item $-5$ \quad \item $5$ \quad \item $19$
    \end{enumerate}
    \item $-27 + \frac{3}{5}$
    \begin{enumerate}
        \item $-\frac{31}{5}$ \quad \item $-\frac{11}{5}$ \quad \item $\frac{11}{5}$ \quad \item $\frac{31}{5}$
    \end{enumerate}
    \item $\frac{54a^2b \times 3b}{9ab}$
    \begin{enumerate}
        \item $2ab$ \quad \item $2a^2b$ \quad \item $18ab$ \quad \item $18a^2b$
    \end{enumerate}
    \item $\frac{3x - y}{2} - \frac{2x + 4y}{3}$
    \begin{enumerate}
        \item $\frac{5x - 11y}{6}$ \quad \item $\frac{5x - 5y}{6}$ \quad \item $\frac{13x - 11y}{6}$ \quad \item $\frac{13x - 5y}{6}$
    \end{enumerate}
    \item $(3+\sqrt{5})(3-\sqrt{5}) - 6(1 - \sqrt{5})$
    \begin{enumerate}
        \item $-8 + 6\sqrt{5}$ \quad \item $-2 + 6\sqrt{5}$ \quad \item $-8 + 12\sqrt{5}$ \quad \item $-2 + 12\sqrt{5}$
    \end{enumerate}
\end{enumerate}

\section*{問2}
次の問いに対する答えとして正しいものを、それぞれ1~4の中から選びなさい。

\begin{enumerate}
    \item $(x - 3)^2 - 4(x - 3) - 32$ を因数分解しなさい。
    \begin{enumerate}
        \item $(x - 11)(x + 1)$ \quad \item $(x - 8)(x + 4)$ \quad \item $(x + 8)(x - 4)$ \quad \item $(x + 11)(x - 1)$
    \end{enumerate}
    \item 二次方程式 $4x^2 + 6x + 1 = 0$ を解きなさい。
    \begin{enumerate}
        \item $x = \frac{-3 \pm \sqrt{5}}{4}$ \quad \item $x = \frac{-3 \pm \sqrt{13}}{4}$ \quad \item $x = \frac{3 \pm \sqrt{5}}{4}$ \quad \item $x = \frac{3 \pm \sqrt{13}}{4}$
    \end{enumerate}
    \item 関数 $y = ax^2$ について、$x$ の値が $2$ から $4$ まで増加するときの変化の割合が $4$ であった。このときの $a$ の値を求めなさい。
    \begin{enumerate}
        \item $a = \frac{1}{4}$ \quad \item $a = \frac{1}{3}$ \quad \item $a = \frac{1}{2}$ \quad \item $a = \frac{2}{3}$
    \end{enumerate}
    \item 大,小2つの正方形があり、大きい正方形の1辺の長さは小さい正方形の1辺の長さより $9$ cm 長く、2つの正方形の面積の和は $305$ cm$^2$ であった。このとき、小さい正方形の1辺の長さを求めなさい。
    \begin{enumerate}
        \item $3$ cm \quad \item $5$ cm \quad \item $7$ cm \quad \item $9$ cm
    \end{enumerate}
    \item 半径が $9$ cm、弧の長さが $2\pi$ cm のおうぎ形の面積を求めなさい。
    \begin{enumerate}
        \item $6\pi$ cm$^2$ \quad \item $9\pi$ cm$^2$ \quad \item $18\pi$ cm$^2$ \quad \item $27\pi$ cm$^2$
    \end{enumerate}
    \item $360n$ が整数となるような正の整数 $n$ の個数を求めなさい。
    \begin{enumerate}
        \item 3個 \quad \item 4個 \quad \item 5個 \quad \item 6個
    \end{enumerate}
\end{enumerate}


\section*{問3}
次の問いに答えなさい。

(ア) 右の図1のように、円Oの周上に異なる3点 A, B, C を $AB < AC$ で、$\angle ABC$ が鋭角となるようにとり、点 A を含まない BC 上に点 D を、$\angle ABC = \angle ACD$ となるようにとる。

また、線分 AC 上に点 E を、$DC \parallel BE$ となるようにとり、線分 AD と線分 BE との交点を F とする。

このとき、次の(イ)、(ウ)に答えなさい。

(i) $\triangle ABF$ と $\triangle BCE$ が相似であることを次のように証明した。$\mathbf{(a)}$, $\mathbf{(b)}$ に最も適するものをそれぞれ選択肢の1~4の中から1つずつ選びなさい。

\textbf{証明}
\begin{itemize}
    \item $\triangle ABF$ と $\triangle BCE$ において、
    \item まず、$BD$ に対する円周角は等しいから、
    \[ \mathbf{(a)} \]
    \item また、$DC \parallel BE$ より、平行線の錯角は等しいから、
    \[ \angle BCD = \angle CBE \]
    \item よって、$\angle BAF = \angle CBE$ \quad \text{(1)}
    \item 次に、仮定より、
    \[ \angle ABC = \angle ACD \]
    \item よって、
    \[ \angle ABE = \angle ABC - \angle CBE \]
    \[ = \angle ACD - \angle BCD \]
    \item これより、$\angle ABF = \angle ACB$ \quad \text{(2)}
    \item (1), (2) より、$\mathbf{(b)}$ から、
    \[ \triangle ABF \sim \triangle BCE \]
\end{itemize}

(a) の選択肢
\begin{enumerate}
    \item $\angle ABC = \angle ADC$
    \item $\angle ACD = \angle AEB$
    \item $\angle ABE = \angle BCD$
    \item $\angle BED = \angle CDE$
\end{enumerate}

(b) の選択肢
\begin{enumerate}
    \item 1組の辺とその両端の角がそれぞれ等しい
    \item 2組の辺の比とその間の角がそれぞれ等しい
    \item 3組の辺の比がすべて等しい
    \item 2組の角がそれぞれ等しい
\end{enumerate}

(ii) 次の $\mathbf{(あ)}$, $\mathbf{(い)}$, $\mathbf{(う)}$, $\mathbf{(え)}$, $\mathbf{(お)}$ にあてはまる数字をそれぞれ0~9の中から1つずつ選びなさい。

線分 $BC$ と線分 $DE$ の交点を $G$ とする。$AE=9$ cm, $CD=8$ cm, $DF=3$ cm のとき、三角形 $CEG$ の面積は
\[ \mathbf{(あ)} \mathbf{(い)} \mathbf{(う)} \]
\[ \mathbf{(え)} \mathbf{(お)} \] cm$^2$ である。

(イ) K さんは、ある中学校の3年生で、サッカー部に所属している。右の図2は、サッカー部に所属する3年生20人それぞれの、サッカーの経験年数をヒストグラムに表したものである。なお、階級はいずれも、1年以上2年未満、2年以上3年未満などのように、階級の幅を1年にとって分けている。

 放課後に1人10本ずつ20人全員がシュートの練習を行い、それぞれのシュートの成功した数を記録することになった。

 K さんは、サッカー部の3年生を、経験年数3年未満の生徒と3年以上の生徒の2つのグループに分け、シュートの成功した数を比較することにした。

 次の資料は、経験年数3年未満の生徒と3年以上の生徒について、それぞれのシュートの成功した数をKさんが調べて記録したものである。

\textbf{資料 (単位:本)}
\begin{itemize}
    \item 経験年数3年未満の生徒: 5, 4, 2, 9, 5, 3, 5, 6, 10
    \item 経験年数3年以上の生徒: 4, 3, 4, 5, 8, 8, 6, 8, 3, 5, 9
\end{itemize}

\textbf{(i)}
サッカーの経験年数の中央値が含まれる階級として正しいものを、次の1~4の中から1つ選び、その番号を答えなさい。

\begin{enumerate}
    \item 1年以上2年未満
    \item 2年以上3年未満
    \item 3年以上4年未満
    \item 4年以上5年未満
\end{enumerate}

\textbf{(ii)}
K さんは、資料から読み取ったことを次のようにまとめた。 $\mathbf{(a)}$ 、 $\mathbf{(b)}$ にあてはまるものの組み合わせとして最も適するものを、次の1~9の中から1つ選び、その番号を答えなさい。

\textbf{資料からわかったこと}
\begin{itemize}
    \item 経験年数3年以上の生徒のほうが、3年未満の生徒よりも $\mathbf{(a)}$ がどちらも大きい。
    \item 経験年数3年以上の生徒のほうが、3年未満の生徒よりも $\mathbf{(b)}$ 。
\end{itemize}

\begin{enumerate}
  \item (a): 平均値と最頻値, (b): 最大値と最小値がどちらも大きい
  \item (a): 平均値と最頻値, (b): 第1四分位数と第3四分位数がどちらも大きい
  \item (a): 平均値と最頻値, (b): 成功した数が5本以上の生徒の割合が大きい
  \item (a): 最頻値と中央値, (b): 最大値と最小値がどちらも大きい
  \item (a): 最頻値と中央値, (b): 第1四分位数と第3四分位数がどちらも大きい
  \item (a): 最頻値と中央値, (b): 成功した数が3本以下の生徒の割合が小さい
  \item (a): 中央値と平均値, (b): 第1四分位数と第3四分位数がどちらも大きい
  \item (a): 中央値と平均値, (b): 成功した数が5本以上の生徒の割合が大きい
  \item (a): 中央値と平均値, (b): 成功した数が3本以下の生徒の割合が小さい
\end{enumerate}

\textbf{(ウ)}
右の図3のような、$AB = CD = 10$ cm、$AD = 6$ cm、$BC = 22$ cm、$AD \parallel BC$ の台形 $ABCD$ があり、この台形の辺上を動く点 P がある。

点 P は毎秒 $1$ cm の速さで、点 A を出発して辺 AD を通って点 C まで動き、点 C に着いたところで止まる。

このとき、次の(i)、(ii)に答えなさい。

(i) 点 P が点 A を出発してから $4$ 秒後の、三角形 $ABP$ の面積として正しいものを、次の1~6の中から1つ選び、その番号を答えなさい。

\begin{enumerate}
    \item $12 \text{cm}^2$
    \item $14 \text{cm}^2$
    \item $16 \text{cm}^2$
    \item $18 \text{cm}^2$
    \item $20 \text{cm}^2$
\end{enumerate}

(ii) 点 P が点 A を出発してから $x$ 秒後の、三角形 $ABP$ の面積を $y$ cm$^2$ とする。点 P が辺 DC 上を動くときの、$x$ と $y$ の関係を式で表したものとして正しいものを、次の1~6の中から1つ選び、その番号を答えなさい。

\begin{enumerate}
    \item $y = 3x$
    \item $y = \frac{5}{2} x + 5$
    \item $y = \frac{18}{5} x + 5$
    \item $y = \frac{24}{5} x - 54$
    \item $y = \frac{24}{5} x + 18$
\end{enumerate}

(エ) 次の□の中の「か」「き」にあてはまる数字をそれぞれ0~9の中から1つずつ選び、その数字を答えなさい。

右の図4において、四角形 $ABCD$ は $AD \parallel BC$、$\angle BCD = 90^\circ$ の台形であり、点 E は線分 AC の中点、$\angle CDE = 34^\circ$ である。

また、点 F は辺 AB 上の点で、$AD = AF$、AB ⊥ CF である。

このとき、$\angle BEF = \fbox{か\き}^\circ$ である。

\section*{問4}

右の図において、直線①は関数 $y = \frac{1}{6} x + 5$ のグラフであり、直線②は関数 $y = ax^2$ のグラフ、直線③は関数 $y = -\frac{5}{2} x$ のグラフである。

点 A は直線①と直線②の交点で、その $x$ 座標は $−6$ である。点 B は直線①の直線で、線分 AB は y 軸に平行である。点 C は線分 AB 上の点で、$AC:CB = 1:2$ である。

また、2点 D, E は直線③の上にあって、その $x$ 座標はそれぞれ $−5, 4$ である。

また、点 F は直線③と D を結んだ交点であり、点 G は線分 DE 上の点で、$DG = GE$ である。

原点を O とするとき、次の問いに答えなさい。

(ア) 直線の式 $y = ax^2$ の $a$ の値として正しいものを、次の1~6の中から1つ選び、その番号を答えなさい。

\begin{enumerate}
    \item $a = \frac{1}{9}$
    \item $a = \frac{1}{6}$
    \item $a = \frac{2}{9}$
    \item $a = \frac{1}{3}$
    \item $a = \frac{2}{3}$
\end{enumerate}

(イ) 直線 $BF$ の式を $y = mx + n$ とするときの(i) $m$ の値と、(ii) $n$ の値として正しいものを、それぞれ次の1~6の中から1つずつ選び、その番号を答えなさい。

(i) $m$ の値

\begin{enumerate}
    \item $m = \frac{3}{7}$
    \item $m = \frac{5}{7}$
    \item $m = \frac{7}{5}$
    \item $m = \frac{8}{7}$
    \item $m = \frac{7}{3}$
\end{enumerate}

(ii) $n$ の値

\begin{enumerate}
    \item $n = \frac{10}{7}$
    \item $n = \frac{12}{7}$
    \item $n = \frac{14}{5}$
    \item $n = \frac{16}{5}$
    \item $n = \frac{18}{7}$
\end{enumerate}

(ウ) 次の□の中の「く」「け」「こ」にあてはまる数字をそれぞれ0~9の中から1つずつ選び、その数字を答えなさい。

線分 BC 上に点 H を、四角形 $CDGH$ と四角形 $BHGE$ の面積が等しくなるようにとる。このときの、点 H の x 座標は \frac{\fbox{くけ}}{\fbox{こ}} である。

\section*{問5}

右の図1のような、$AB = 5$ cm、$BC = 13$ cm の長方形 $ABCD$ がある。

大、小 2 つのさいころを同時に 1 回投げ、大きいさいころの出た目を $a$、小さいさいころの出た目の数を $b$ とする。出た目の数によって、次の【操作1】、【操作2】を順に行い、長方形 $ABCD$ を 3 つの長方形に分ける。

【操作1】辺 $AD$ 上に点 $E$ を、$AE = (a + b)$ cm となるようにとり、辺 $BC$ 上に点 $F$ を、$AB \parallel EF$ となるようにとり、線分 $EF$ を引く。

【操作2】辺 $AB$ 上に点 $H$ を、線分 $EF$ 上に点 $H$ を、$AE \parallel GH$ で、長方形 $AGHE$ の面積が $abc$ cm$^2$ となるようにとり、線分 $GH$ を引く。

長方形 $AGHE$ を $P$、長方形 $BFHG$ を $Q$、長方形 $CDEF$ を $R$ とし、それぞれの面積について考える。

例えば、大きいさいころの出た目の数が $4$、小さいさいころの出た目の数が $2$ のとき、$a = 4$, $b = 2$ だから、【操作1】により、辺 $AD$ 上に点 $E$ を、$AE = 6$ cm となるようにとり、辺 $BC$ 上に点 $F$ を、$AB \parallel EF$ となるようにとり、線分 $EF$ を引く。

次に、【操作2】により、辺 $AB$ 上に点 $H$ を、線分 $EF$ 上に点 $H$ を、$AE \parallel GH$ で、長方形 $AGHE$ の面積が $8$ cm$^2$ となるようにとり、線分 $GH$ を引く。

この結果、$P, Q, R$ は図2のようになり、$P$ の面積は $8$ cm$^2$、$Q$ の面積は $22$ cm$^2$、$R$ の面積は $35$ cm$^2$ となる。

いま、図1 の状態で、大、小 2 つのさいころを同時に 1 回投げるとき、次の問いに答えなさい。ただし、大、小 2 つのさいころはともに、1 から 6 までの目が出ることも同様に確からしいものとする。

(ア) 次の□の中の「さ」「し」にあてはまる数字をそれぞれ0~9の中から1つずつ選び、その数字を答えなさい。

$P$ の面積が $6$ cm$^2$ となる確率は $\fbox{さし}$ である。

(イ) 次の□の中の「す」「せ」にあてはまる数字をそれぞれ0~9の中から1つずつ選び、その数字を答えなさい。

$P$ の面積と $R$ の面積がどちらも $Q$ の面積より小さくなる確率は $\fbox{すせ}$ である。

\section*{問6}

右の図は、正方形 $ABCD$ を底面とし、点 $E$ を頂点とする正四角すいであり、点 $F$ は辺 $AE$ の中点である。

また、点 $G$ は辺 $AB$ 上の点で、$AG:GB=1:2$ である。

$AB=AE=6$ cm のとき、次の問いに答えなさい。

(ア) この正四角すいの表面積として正しいものを、次の1~6の中から1つ選び、その番号を答えなさい。

\begin{enumerate}
    \item $(18+6\sqrt{3})$ cm$^2$
    \item $(18+18\sqrt{3})$ cm$^2$
    \item $(18+36\sqrt{3})$ cm$^2$
    \item $(36+6\sqrt{3})$ cm$^2$
    \item $(36+18\sqrt{3})$ cm$^2$
    \item $(36+36\sqrt{3})$ cm$^2$
\end{enumerate}

(イ) 次の□の中の「そ」「た」「ち」「つ」にあてはまる数字をそれぞれ0~9の中から1つずつ選び、その数字を答えなさい。

この正四角すいにおいて、3点 $C, F, G$ を結んでできる三角形の面積は $\fbox{そたち}$ cm$^2$ である。


\end{document}
