\documentclass{ltjsarticle}
\begin{document}
\title{2018年度 追試験解説}

\section{問1}

(ア) $-13 + 2 = -11$

(イ) $\frac{3}{8} - \frac{3}{5} = \frac{15}{40} - \frac{24}{40} = -\frac{9}{40}$

(ウ) $30 a^{2} b^{2}\div (-6ab) = 30a^{2}b^{2} \times (-\frac{1}{6ab}) = -5ab$

(エ) $-\frac{25}{\sqrt{5}} + \sqrt{20} = -\frac{25\sqrt{5}}{5} + 2\sqrt{5} = -5\sqrt{5} + 2\sqrt{5} = -3\sqrt{5}$

(オ) $-(x-2)^{2} + (x-8)(x+3) = -(x^{2} - 4x + 4) + (x^{2} - 5x - 24)$

$= -x^{2} + 4x - 4 + x^{2} - 5x - 24$

$= -x - 28$

\section{問2}
(ア) $(x-3)^{2}+5(x-3)-36$を因数分解しなさい。
 \\$ A = (x-3) $ とすると
 \\$(x-3)^{2}+5(x-3)-36 = A^{2} + 5A - 36 = (A+9)(A-4) = (x-3+9)(x-3-4) = (x+6)(x-7)$

(イ) 2次方程式 $ 5x^{2}-8x+1=0 $を解きなさい。
 \\ 解の公式より
 \\ $x=\frac{8\pm\sqrt{(-8)^{2}-4\times5\times1}}{2\times5}$
 \\ $x=\frac{8\pm\sqrt{64-20}}{10}$
 \\ $x=\frac{8\pm\sqrt{44}}{10}$
 \\ $x=\frac{8\pm2\sqrt{11}}{10}$
 \\ $x=\frac{4\pm\sqrt{11}}{5}$

(ウ) $x$の値が$-4$から$-1$まで増加するとき、2つの関数$y=ax^{2}$と$y=-3x$の変化の割合が等しくなるような$a$の値を求めなさい。
 \\ $x=-4$のとき$y=ax^{2}=16a$、$x=-1$のとき$y=ax^{2}=a$
 \\ このとき変化の割合は
 \\ $\frac{a - 16a}{-1 - (-4)} = \frac{-15a}{3} = -5a$
 \\ これは$y=-3x$の変化の割合$-3$と等しいので
 \\ $-5a = -3$
 \\ $a = \frac{3}{5}$

(エ) A商店では, ある品物を仕入れたときの値段に対して50\%増しの価格をつけたが売れなかったので, その価格の20\%引きで売ることにしたところ、割引き後の価格は仕入れたときの値段よりも120円高くなった。この品物を仕入れたときの値段を求めなさい。
 \\ 仕入れたときの値段を$x$円とすると、50\%増しの価格は$1.5x$円となる。この価格を20\%引きすると$0.8\times1.5x=1.2x$円となる。この価格は$x+120$円となるので
 \\ $1.2x = x + 120$
 \\ $0.2x = 120$
 \\ $x = 600$

(オ) 50Lの水が入った水そうから毎分$a$Lずつ水を減らしていったところ, 5分後に, 水そう水は$20$L以上残っていた。このときの数量を不等式で表しなさい。
 \\ 1分間で減る水の量は$a$Lなので、5分後には$-5a$L減っている。したがって残っている水の量は$50-5a$であり、これは$20$以上であるので
 \\ $50-5a \geq 20$

(カ) $\sqrt{\frac{720}{n}}$が整数となるような正の整数$n$の個数を求めなさい。

\end{document}